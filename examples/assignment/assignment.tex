\documentclass{assignmeownt}

\coursenumber{CAT 101}
\coursetitle{Introduction to the Religion of Cats}
\doctitle{Homework 1}
\docauthor{Meowthor}

\begin{document}
\maketitle
\thispagestyle{firststyle}

\question
\questionpart{What sounds does a cat make?}
\par Cats can make a `meow', `purr', or `boioioing' sound.

\bigskip

\todo{Add more sounds before the cat overlords punish me.}

\questionpart{Prove that cats are Gods.}
We must prove that cats are Gods.

\begin{proof}
  \par Let us assume for the sake of contradiction that cats are not Gods.
  \par This must mean that the world we are living in cannot exist.
  \par But the world we are living in does exist.
  \par This must mean that our initial assumption that cats are not Gods must be incorrect.
\end{proof}

\par \finalresult{Therefore, cats are Gods.}

\question{Prove that cats are equal to infinity.}
\par We must first prove that cats are a number.
\begin{lemma}
  Cats are a number.
\end{lemma}
\begin{proof}
  \par Obviously, cats are a number.
\end{proof}

\par Since cats are a number, we can now prove that cats are equal to infinity.
\begin{proof}
  Let cat be \( x \).
  \begin{flalign}
             & x = x - 1  &  & \text{(Duh)} \nonumber \\
    \implies & x + 1 = x  &  & \nonumber              \\
    \implies & x = \infty &  & \nonumber
  \end{flalign}

  \par We can clearly see that \( x = \infty \).
\end{proof}

\par \finalresult{Therefore, cats are equal to infinity.}
\end{document}
